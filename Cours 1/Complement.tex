\makeatletter\let\ifGm@compatii\relax\makeatother
\documentclass{beamer}
\usepackage{etex}

\usepackage[utf8]{inputenc}
\usepackage[T1]{fontenc}
\usepackage[francais, turkish]{babel}
\usepackage{amssymb,amsthm,amsmath}
\usepackage{xcolor}
\usepackage{tikz}
\usepackage[all]{xy}

\usepackage{listings}
%\usepackage{verbatim}

\usepackage{bera}
\usepackage{caption}
\captionsetup{%
   margin=0em,
   font={scriptsize,rm},
   labelfont={color=Maroon,bf},
   justification=justified,%default: RaggedRight. Other options: justified, centering
   labelsep=quad%default:colon. Options: period, space, quad, newline
}
\usepackage{showexpl}
\usepackage{etoolbox}
\makeatletter
\patchcmd{\SX@codeInput}{xleftmargin=0pt,xrightmargin=0pt}{}
  {\typeout{***Successfully patched \protect\SX@codeInput***}}
  {\typeout{***ERROR! Failed to patch \protect\SX@codeInput***}}
\makeatother

\lstloadlanguages{[LaTeX]Tex} 
\lstset{% 
    basicstyle=\ttfamily\small, 
    commentstyle=\itshape\ttfamily\small, 
    showspaces=false, 
    showstringspaces=false, 
    breaklines=true, 
    breakautoindent=true, 
    captionpos=t 
} 
\lstset{%
    breaklines=true,%default : false 
    breakindent=10pt,%default: 20pt 
    linewidth=\linewidth,%default : \linewidth,
    basicstyle=\ttfamily\tiny,% cannot take arguments
    keywordstyle=\color{Blue}\sffamily\bfseries,                                
    identifierstyle=\color{Black},                                      
    commentstyle=\color{OliveGreen}\itshape,                                    
    stringstyle=\rmfamily,                                                      
    showspaces=false,%default false
    showstringspaces=false,%default: true
    backgroundcolor=\color{Yellow!30},
    frame=single,%default frame=none 
    rulecolor=\color{Red},  
    % the following must be defined to make hacking work.
    framerule=0.4pt,%expand outward 
    framesep=3pt,%expand outward
    xleftmargin=3.4pt,%to make the frame fits in the text area. 
    xrightmargin=3.4pt,%to make the frame fits in the text area. 
    tabsize=2%,%default :8 only influence the lstlisting and lstinline.
}
\lstset{%
    %explpreset below will affect LTX only.
    explpreset={%
                            language={[LaTeX]TeX},
                            captionpos=t,
                            aboveskip=2\fboxsep,                        
                            pos=b,
                            vsep=\fboxsep%vertical space between formatted and input.
                         }%
}

\usetheme{Hannover}

\newtheorem{defi}{Définition}
\newtheorem{thm}{Théorème}

\title{Packages \LaTeX}
\author{Charles Martin}

\begin{document}

\selectlanguage{francais}

\definecolor{Red}{RGB}{255,0,0}
\definecolor{Grey}{RGB}{205,201,201}
\definecolor{Yellow}{RGB}{255,255,0}
\definecolor{Blue}{RGB}{50,50,150}
\definecolor{Black}{RGB}{0,0,0}
\definecolor{OliveGreen}{RGB}{0,100,0}

\frame{\titlepage}

\section{Packages}
\subsection{Packages courants}
\frame[containsverbatim]{
\frametitle{Les packages}
La commande usepackage spécifie l'utilisation d'un package.

\textbackslash usepackage[options]{nom du package}

Quelques packages utilisé couramment :

   \begin{itemize}
    \item \textbackslash usepackage[francais, turkish]\{babel\}
    \item \textbackslash usepackage\{tikz\}
    \item \textbackslash usepackage[all]\{xy\}
    \item \textbackslash usepackage\{amsmath\}
    \item \textbackslash usepackage\{graphicx\}
   \end{itemize}
}

\subsection{Babel}
\frame[containsverbatim]{
\frametitle{Babel}
Babel est un package qui permet de générer des documents en plusieurs langues.
\begin{LTXexample}
\selectlanguage{turkish}
Bu T{\"{u}}rk\c cedir,

\selectlanguage{francais}
Et c'est du fran\c cais
\end{LTXexample}

En savoir plus :

\url{http://www.tuteurs.ens.fr/logiciels/latex/manuel.html}
}

\subsection{Tikz}
\frame[containsverbatim]{
\frametitle{Tikz}
TikZ permet de dessiner des schémas.

\begin{LTXexample}
\begin{tikzpicture}
\draw (0,0) circle (1) ;
\end{tikzpicture}
\end{LTXexample}

Remarque : il est aisé de générer des schémas TikZ avec Inkscape.

En savoir plus :

\url{http://math.et.info.free.fr/TikZ/bdd/TikZ-Impatient.pdf}

\url{http://www.texample.net/tikz/examples/}
}

\subsection{xy}
\frame[containsverbatim]{
\frametitle{xy}
xy permet d'afficher des diagrammes.
\begin{LTXexample}
\begin{displaymath}
    \xymatrix{ A \ar[r] & B \ar[d] \\
               D \ar[u] & C \ar[l] }
\end{displaymath}
\end{LTXexample}
}

\frame[containsverbatim]{
\frametitle{xy}
Un exemple plus complexe :
\begin{LTXexample}
\xymatrix{
    A \ar@{^{(}->}[r]^f \ar@{.>}[d]_\phi \ar@{=>}[rd] & B \ar@{=}[d]^\psi \\
    C \ar@{->>}[r]_g & D
  }
\end{LTXexample}
}

\frame[containsverbatim]{
\frametitle{xy}
Un dernier exemple :
\begin{LTXexample}
\xymatrix{
  A \ar[r]^u_d \ar[rd]^u_d &
  B \ar[d]^u_d &
  C \ar[l]^u_d \ar[ld]^u_d\\
  &D}
\end{LTXexample}

Pour en savoir plus : 

\url{http://math.arizona.edu/~swig/documentation/xypic/Xypic.pdf}
}

\subsection{amsmath}
\frame[containsverbatim]{
\frametitle{amsmath}
amsmath fournit de nombreuses fonctionnalités pour afficher des équations ou des structures mathématiques.
\begin{LTXexample}
\begin{align}
a_1& =b_1+c_1\\
a_2& =b_2+c_2-d_2+e_2
\end{align}
\end{LTXexample}

Pour en savoir plus : 

\url{http://mirror.isoc.org.il/pub/ctan/macros/latex/required/amslatex/amscls/doc/amsthdoc.pdf}
}

\frame[containsverbatim]{
\frametitle{amsmath}
amsmath permet également de générer des matrices.
\begin{LTXexample}
$A_{m,n} =
 \begin{pmatrix}
  a_{1,1} & a_{1,2} & \cdots & a_{1,n} \\
  a_{2,1} & a_{2,2} & \cdots & a_{2,n} \\
  \vdots  & \vdots  & \ddots & \vdots  \\
  a_{m,1} & a_{m,2} & \cdots & a_{m,n}
 \end{pmatrix}$
\end{LTXexample}

Pour en savoir plus : 

\url{http://en.wikibooks.org/wiki/LaTeX/Mathematics}
}

\subsection{amsthm}
\frame[containsverbatim]{
\frametitle{amsthm}
amsthm étend la fonction \textbackslash newtheorem.
\begin{LTXexample}
\newtheorem{lem}{Th{\'e}or{\`e}me}
\begin{lem} 
Ceci est mon th{\'e}or{\`e}me. 
\end{lem}
\end{LTXexample}

Pour en savoir plus : 

\url{http://mirror.isoc.org.il/pub/ctan/macros/latex/required/amslatex/amscls/doc/amsthdoc.pdf}
}

\subsection{graphicx}
\frame[containsverbatim]{
\frametitle{graphicx}
graphicx permet d'afficher des images.
\begin{LTXexample}
\begin{figure}[position]
   \includegraphics[scale=0.10]{./Images/latex.jpg}
\end{figure}
\end{LTXexample}

Pour en savoir plus : 

\url{http://fr.wikibooks.org/wiki/LaTeX/Inclure\_des\_images}
}

\section{O{\`u} trouver ces documents}
\frame[containsverbatim]{
	\frametitle{O{\`u} trouver ces documents}
	Vous retrouverez les documents présentés ci-dessous.
	
	Cours de M. Bailly-Bechet : \url{http://pbil.univ-lyon1.fr/members/mbailly/Comm_Scientifique/M1/cours_latex.pdf}
	
	Les exemples d'utilisation de packages :
	\begin{itemize}
		\item Source : \url{https://github.com/MC-Course/LaTeX/blob/master/Cours%201/Complement.tex}
		\item pdf : \url{https://github.com/MC-Course/LaTeX/blob/master/Cours%201/Complement.pdf?raw=true}
	\end{itemize}	
}

\end{document}