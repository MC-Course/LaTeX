\makeatletter\let\ifGm@compatii\relax\makeatother
\documentclass{beamer}
\usepackage{etex}

\usepackage[utf8]{inputenc}
\usepackage[T1]{fontenc}
\usepackage[francais, turkish, spanish]{babel}
\usepackage{amssymb,amsthm,amsmath}
\usepackage{xcolor}
\usepackage{color}
\usepackage{tikz}
\usepackage[all]{xy}
\usepackage{pgfplots}

\usepackage{listings}
%\usepackage{verbatim}

\usepackage{bera}
\usepackage{caption}
\captionsetup{%
   margin=0em,
   font={scriptsize,rm},
   labelfont={color=Maroon,bf},
   justification=justified,%default: RaggedRight. Other options: justified, centering
   labelsep=quad%default:colon. Options: period, space, quad, newline
}
\usepackage{showexpl}
\usepackage{etoolbox}
\makeatletter
\patchcmd{\SX@codeInput}{xleftmargin=0pt,xrightmargin=0pt}{}
  {\typeout{***Successfully patched \protect\SX@codeInput***}}
  {\typeout{***ERROR! Failed to patch \protect\SX@codeInput***}}
\makeatother

\lstloadlanguages{[LaTeX]Tex} 
\lstset{% 
    basicstyle=\ttfamily\small, 
    commentstyle=\itshape\ttfamily\small, 
    showspaces=false, 
    showstringspaces=false, 
    breaklines=true, 
    breakautoindent=true, 
    captionpos=t 
} 
\lstset{%
    breaklines=true,%default : false 
    breakindent=10pt,%default: 20pt 
    linewidth=\linewidth,%default : \linewidth,
    basicstyle=\ttfamily\tiny,% cannot take arguments
    keywordstyle=\color{Blue}\sffamily\bfseries,                                
    identifierstyle=\color{Black},                                      
    commentstyle=\color{OliveGreen}\itshape,                                    
    stringstyle=\rmfamily,                                                      
    showspaces=false,%default false
    showstringspaces=false,%default: true
    backgroundcolor=\color{Yellow!30},
    frame=single,%default frame=none 
    rulecolor=\color{Red},  
    % the following must be defined to make hacking work.
    framerule=0.4pt,%expand outward 
    framesep=3pt,%expand outward
    xleftmargin=3.4pt,%to make the frame fits in the text area. 
    xrightmargin=3.4pt,%to make the frame fits in the text area. 
    tabsize=2%,%default :8 only influence the lstlisting and lstinline.
}
\lstset{%
    %explpreset below will affect LTX only.
    explpreset={%
                            language={[LaTeX]TeX},
                            captionpos=t,
                            aboveskip=2\fboxsep,                        
                            pos=b,
                            vsep=\fboxsep%vertical space between formatted and input.
                         }%
}

\usetheme{Hannover}

\newtheorem{defi}{Définition}
\newtheorem{thm}{Théorème}

\title{\LaTeX - exercices}
\author{Charles Martin}

\begin{document}

\selectlanguage{francais}

\definecolor{Red}{RGB}{255,0,0}
\definecolor{Grey}{RGB}{205,201,201}
\definecolor{Yellow}{RGB}{255,255,0}
\definecolor{Blue}{RGB}{50,50,150}
\definecolor{Black}{RGB}{0,0,0}
\definecolor{OliveGreen}{RGB}{0,100,0}

\frame{
	\titlepage
	Inspiré du cours de Marc Bailly-Bechet. \cite{cours}
}

\frame[containsverbatim]{
	\frametitle{Template}
	Les documents nécessaires aux exercices : \newline
	
	\begin{center}
		Cours : \url{http://tiny.cc/LatexCours}
		
		Template : \url{http://tiny.cc/LatexMinimal}
		
		Exercices : \url{http://tiny.cc/LatexExercices}
	\end{center}
}

\section{Formules}
\frame[containsverbatim]{
	\frametitle{Exercice 1}
	
	Reproduisez le modèle ci-dessous en \LaTeX:
	\begin{align*}
		a^2 + b^2 + 2ab	&<= 2(a^2 + b^2) \\
		(a + b)^2		&<= 2(a^2 + b^2) \\
		\left(\frac{a + b}{2}\right)^2		&<= \frac{a^2 + b^2}{2}
	\end{align*}
}

\section{Tikz}
\frame[containsverbatim]{
	\frametitle{Exercice 2.1}
	
	Reproduisez le modèle ci-dessous en \LaTeX:
	
	\begin{tikzpicture}
	\begin{axis}[legend pos=outer north east]
	\addplot[color=blue]{x};
	\addplot[color=red]{x^2};
	\addplot[color=green]{x^3};
	\legend{$x$,$x^2$,$x^3$}
	\end{axis}
	\end{tikzpicture}
}

\frame[containsverbatim]{
	\frametitle{Exercice 2.2}
	
	Reproduisez le modèle ci-dessous en \LaTeX:
	
	\begin{tikzpicture}
	\begin{axis}[domain=0:1,legend pos=outer north east]
	\addplot {cos(deg(x))}; 
	\addplot {sin(deg(x))};
	\addplot {tan(deg(x))};
	\addplot {x^2};
	\addplot {x^3};
	\legend{$\cos(x)$,$\sin(x)$,$\tan(x)$,$x^2$,$x^3$}
	\end{axis}
	\end{tikzpicture}
}

\section{Tableau}
\frame[containsverbatim]{
	\frametitle{Exercice 3}
	
	Reproduisez le modèle ci-dessous en \LaTeX:
	
	\begin{tabular}{|*{2}{c|}l r|}
		\hline
		case 1 & case 2 & case 3 & case 4 \\
		centrée & centrée & gauche & droite \\
		\hline
		pied 1 & pied 2 & pied 3 & pied 4 \\
		\hline
	\end{tabular}
}

\section{xy}
\frame[containsverbatim]{
	\frametitle{Exercice 4.1}
	
	Reproduisez le modèle ci-dessous en \LaTeX:
	
	\begin{displaymath}
	\xymatrix{
		T_1 \ar@/_/[rdd] \ar@{=>}[rd] \ar@/^/[rrrd] \ar@/^-6pc/[rrdd] \\
		& A \ar[r] \ar[d] & B \ar[d] \ar@{-->}[r] & E \ar[ld] \\
		& C \ar[ru] & D  \ar[l] \ar[lu]
	}
	\end{displaymath}
}

\frame[containsverbatim]{
	\frametitle{Exercice 4.1}
	
	Reproduisez le modèle ci-dessous en \LaTeX:
	
%	\begin{displaymath}
%		 \xymatrix{
%		 	A \ar@/^/@{^{<}-_{>}[rr] & B}
%		 	}
%	\end{displaymath}
}

\end{document}