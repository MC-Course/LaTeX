%TODO : Changer la section des packages en section par fonctionnalité (ajouter une image, un schéma, etc.)


%TODO : Compléter en se basant sur les cours tels que :
%http://pbil.univ-lyon1.fr/members/mbailly/Comm_Scientifique/M1/cours_latex.pdf
%http://pbil.univ-lyon1.fr/members/mbailly/Comm_Scientifique/
%http://tex.loria.fr/general/apprends-latex.pdf

%http://www.math.ens.fr/~millien/tdlatex/poly.pdf

%Qui suis-je ?
%Latex, les éléments clés
%Latex, les packages

\makeatletter\let\ifGm@compatii\relax\makeatother
\documentclass{beamer}
\usepackage{etex}

\usepackage[utf8]{inputenc}
\usepackage[T1]{fontenc}
\usepackage[francais, turkish]{babel}
\usepackage{amssymb,amsthm,amsmath}
\usepackage{xcolor}
\usepackage{color}
\usepackage{tikz}
\usepackage[all]{xy}

\usepackage{listings}
%\usepackage{verbatim}

\usepackage{bera}
\usepackage{caption}
\captionsetup{%
   margin=0em,
   font={scriptsize,rm},
   labelfont={color=Maroon,bf},
   justification=justified,%default: RaggedRight. Other options: justified, centering
   labelsep=quad%default:colon. Options: period, space, quad, newline
}
\usepackage{showexpl}
\usepackage{etoolbox}
\makeatletter
\patchcmd{\SX@codeInput}{xleftmargin=0pt,xrightmargin=0pt}{}
  {\typeout{***Successfully patched \protect\SX@codeInput***}}
  {\typeout{***ERROR! Failed to patch \protect\SX@codeInput***}}
\makeatother

\lstloadlanguages{[LaTeX]Tex} 
\lstset{% 
    basicstyle=\ttfamily\small, 
    commentstyle=\itshape\ttfamily\small, 
    showspaces=false, 
    showstringspaces=false, 
    breaklines=true, 
    breakautoindent=true, 
    captionpos=t 
} 
\lstset{%
    breaklines=true,%default : false 
    breakindent=10pt,%default: 20pt 
    linewidth=\linewidth,%default : \linewidth,
    basicstyle=\ttfamily\tiny,% cannot take arguments
    keywordstyle=\color{Blue}\sffamily\bfseries,                                
    identifierstyle=\color{Black},                                      
    commentstyle=\color{OliveGreen}\itshape,                                    
    stringstyle=\rmfamily,                                                      
    showspaces=false,%default false
    showstringspaces=false,%default: true
    backgroundcolor=\color{Yellow!30},
    frame=single,%default frame=none 
    rulecolor=\color{Red},  
    % the following must be defined to make hacking work.
    framerule=0.4pt,%expand outward 
    framesep=3pt,%expand outward
    xleftmargin=3.4pt,%to make the frame fits in the text area. 
    xrightmargin=3.4pt,%to make the frame fits in the text area. 
    tabsize=2%,%default :8 only influence the lstlisting and lstinline.
}
\lstset{%
    %explpreset below will affect LTX only.
    explpreset={%
                            language={[LaTeX]TeX},
                            captionpos=t,
                            aboveskip=2\fboxsep,                        
                            pos=b,
                            vsep=\fboxsep%vertical space between formatted and input.
                         }%
}

\usetheme{Hannover}

\newtheorem{defi}{Définition}
\newtheorem{thm}{Théorème}

\title{Introduction au \LaTeX}
\author{Charles Martin}

\begin{document}

\selectlanguage{francais}

\definecolor{Red}{RGB}{255,0,0}
\definecolor{Grey}{RGB}{205,201,201}
\definecolor{Yellow}{RGB}{255,255,0}
\definecolor{Blue}{RGB}{50,50,150}
\definecolor{Black}{RGB}{0,0,0}
\definecolor{OliveGreen}{RGB}{0,100,0}

\frame{
	\titlepage
	Inspiré du cours de Marc Bailly-Bechet. \cite{cours}
}

\section{\TeX}	
\subsection{Introduction}
\begin{frame}{\TeX}
  Le moteur derrière \LaTeX{} est \TeX{} .
  
  
  \TeX{} a été crée en 1978 par Donald Knuth.
  \begin{block}{Les objectifs de \TeX :}    
   \begin{itemize}
    \item Permettre d’écrire des livres de qualité avec un minimum d’effort
    \item Fournir un système qui donnera toujours le même résultat, maintenant et à l’avenir.
   \end{itemize}
  \end{block}
\end{frame}

\subsection{Syntaxe}
\begin{frame}{Hello World}
	Exemple de Hello World en \TeX{} :
    \lstinputlisting{Sources/HelloWorldTex.tex}
\end{frame}

\section{\LaTeX}
\subsection{Introduction}
\begin{frame}{\LaTeX}
  \LaTeX{} a été crée en 1980 par Leslie Lamport.
  \begin{block}{Les objectifs de \LaTeX :}    
   \begin{itemize}
    \item Fournir un language de haut niveau.
    \item Utiliser la puissance de \TeX{}.
   \end{itemize}
  \end{block}
\end{frame}

\subsection{Pourquoi utiliser \LaTeX ?}
\begin{frame}{Pour quelles raisons choisir \LaTeX ?}
	\begin{block}{Pourquoi utiliser \LaTeX{} plutôt qu'un outil comme Word ?}
		\begin{itemize}
			\item Pour se concentrer sur le contenu. \LaTeX{} s'occupe de la mise forme.
			\item Pour produire des documents de qualité professionnelle.
			\item Pour produire des documents qui produiront toujours le même contenu.
		\end{itemize}
	\end{block}
\end{frame}

\begin{frame}{Que vas-t-on faire avec \LaTeX ?}
	\begin{block}{Quelle sont les utilisations possibles de \LaTeX{} ?}
		\begin{itemize}
			\item Rédiger un rapport de stage, un mémoire, une thèse,
			\item Rédiger document de présentation type powerpoint,
			\item Écrire un livre, une revue, etc.
		\end{itemize}
	\end{block}
\end{frame}

\subsection{Les outils}
\begin{frame}{Installer \LaTeX}
	\begin{block}{Comment installer \LaTeX{} ?}
		\begin{itemize}
			\item Sur Windows, installez MikTeX,
			\item sur Mac, installez MacTeX,
			\item sur Linux, installez le paquet textlive-full.
		\end{itemize}
	\end{block}
\end{frame}

\begin{frame}{Les {\'e}diteurs \LaTeX{}}
	\begin{block}{Les {\'e}diteurs \LaTeX{}}
		Il existe de nombreux {\'e}diteurs \LaTeX.
		
		Parmi les plus connus :
		\begin{itemize}
			\item TexMaker
			\item TeXstudio
			\item Gummi
			\item TeXnicCenter
		\end{itemize}
		
		Ils sont souvent multi-plateforme.
	\end{block}
\end{frame}

\subsection{Les fichiers}
\begin{frame}{Les fichiers}
	
	Génération d'une publication via \LaTeX:
	\begin{displaymath}
	\xymatrix{ 
		.tex \ar@{-->}[rd] & & & .pdf  \\
		.tex \ar@{-->}[r] & .tex \ar[r] & .dvi \ar@{=>}[ru] \ar@{=>}[rd]  \\
		.bib \ar@{-->}[ru] & & & .ps 
	}
	\end{displaymath}
	
	\begin{tabular}{|l|l|}
.tex & Contient les source \\
.dvi & Fichier de compilation intermédiaire \\
.pdf / .ps & Fichier publié à partir du fichier dvi \\
.bib / .ps & Contient la bibliographie
	\end{tabular}
\end{frame}

\subsection{Syntaxe}
\frame[containsverbatim]{
\frametitle{Hello World}
Exemple de Hello World en \LaTeX :
\begin{LTXexample}
\documentclass{article}
\begin{document}
Hello, World !
\end{document}
\end{LTXexample}
}

\frame[containsverbatim]{
\frametitle{Les espaces}
\begin{LTXexample}
Le nombre       d'espaces n'est pas important.
\end{LTXexample}
\begin{LTXexample}
Un paragraphe d{\'e}bute ici.
Une ligne vide apparait.

Un nouveau paragraphe commence.
\end{LTXexample}
}

\frame[containsverbatim]{
\frametitle{Les caractères spéciaux}
Les symboles suivants sont des caractères réservés :

\# \$ \% \^{} \& \_ \{ \} \~{} \textbackslash

Pour les afficher dans votre document, placez un \textbackslash {} devant:

\begin{LTXexample}
\# \$ \% \^{} \& \_ \{ \} \~{}
\textbackslash
\end{LTXexample}
}

\frame[containsverbatim]{
	\frametitle{D{\'e}coupe d'un document}
	Un document \LaTeX{} sera découpé en partie de la manière suivante :
	
	\begin{lstlisting}
	\part{Partie 1}
	\chapter{Chapitre 1}
	\section{Section 1}
	\subsection{Sous-Section 1}
	\subsubsection{Sous-Sous-Section 1}
	\paragraph{Paragraphe 1}
	\subparagraph{Sous-Paragraphe 1}
	\end{lstlisting}
	
	La numérotation est automatique. Il est possible de la désactiver en rajouter *.
	
	\begin{lstlisting}
	\section{Section avec numerotation}
	\section*{Section sans numerotation}
	\end{lstlisting}
}

\frame[containsverbatim]{
	\frametitle{Taille des caractères}
	Il est possible d'ajuster la taille des caractères, avec les commandes suivantes, par taille croissante :
	
	\textbackslash tiny, \textbackslash scriptsize, \textbackslash footnotesize, \textbackslash small,  \textbackslash normalsize, \textbackslash large, \textbackslash Large, \textbackslash LARGE, \textbackslash huge, \textbackslash Huge
	
	\begin{LTXexample}
		\tiny Ce \scriptsize{texte} \footnotesize grossit \small {\`a} \normalsize vue \large d' \Large{oeil.} \\ \LARGE Fuyez \huge pauvre \Huge fou !
	\end{LTXexample}
}

\frame[containsverbatim]{
	\frametitle{Style des caractères}
	De la même manière, il possible d'ajuster le style des caractères :
	
	\textbackslash textbf, \textbackslash textit (ou \textbackslash emph), \textbackslash underline, \textbackslash texttt
	
	\begin{LTXexample}
		\textbf{Le style} \textit{doit} \underline{{\^e}tre} \texttt{utilis{\'e} avec} finesse.
	\end{LTXexample}
}


\frame[containsverbatim]{
	\frametitle{La couleur}
	Ajuster la couleur permettra de mettre en valeur certaines informations.
	
	Pensez à ajouter le bon package : \textbackslash usepackage\{color\}
	
	\begin{LTXexample}
		\color{blue}
		Ce texte est bleu.
		\color{black}
		Ici, on utilise la couleur par d{\'e}faut.
		Et l{\`a}, je met une \textcolor{red}{partie} du texte seulement en rouge.
	\end{LTXexample}
}

\frame[containsverbatim]{
\frametitle{Les formules math{\'e}matiques}
Les formules mat{\'e}matiques sont plac{\'e}es entre \$ et \$.

\begin{LTXexample}
$a^2 + b^2 = c^2$
\end{LTXexample}
}

\frame[containsverbatim]{
\frametitle{Les commandes}
Une commande commence par un \textbackslash{} et a un nom constitué de lettres uniquement.

\textbackslash command[parametre optionnel]\{parametre\}

\begin{LTXexample}
Vous pouvez \textsl{compter} sur moi!
\end{LTXexample}

\begin{LTXexample}
Vous pouvez commencer une nouvelle ligne juste ici ! \newline Merci!
\end{LTXexample}
}

\frame[containsverbatim]{
\frametitle{Les commentaires}
Les commentaires commencent par le caractère \%.
Un commentaire ne sera jamais affiché, et servira généralement à ajouter des remarques sur le code, voir à désactiver du code.
\begin{LTXexample}
Ce message est visible.
%Mais pas ce message.
%\textsl{Ni cette commande.}
\end{LTXexample}
}

\frame[containsverbatim]{
\frametitle{La structure du fichier}
Lorsque \LaTeX{} traite un fichier, il s'attend à suivre une certaine structure.

Cette structure est composée :

   \begin{itemize}
    \item du type de document
    \item du titre
    \item de l'auteur
    \item de la date
    \item du contenu
   \end{itemize}
}

\frame[containsverbatim]{
\frametitle{La structure d'un article}
Exemple de structure d'un article :
\begin{LTXexample}
%On specifie le type de document, ici, un article
\documentclass{article} 
%On specifie le titre de l'article
\title{Document LaTeX} 
%On specifie l'auteur
\author{Charles Martin} 
%On specifie la date
\date{Novembre 2014} 
%Le document commence ici
\begin{document} 
%On trouvera ici le contenu
Contenu du document.
%Le document se termine ici 
\end{document} 
\end{LTXexample}
}

\frame[containsverbatim]{
\frametitle{La structure d'une pr\'esentation type powerpoint}
Exemple de structure d'un article :
\begin{LTXexample}
%On specifie le type de document, ici, un article
\documentclass{beamer} 
%On choisit un theme
\usetheme{Warsaw}
%On specifie le titre de l'article
\title{Pr\'esentation LaTeX} 
%On specifie l'auteur
\author{Charles Martin} 
%On specifie la date
\date{Novembre 2014} 
%Le document commence ici
\begin{document} 
%\begin{frame}
%\frametitle{Titre}
%\framesubtitle{Sous-Titre}
%Contenu
%\end{frame}
\end{document} 
\end{LTXexample}
}

\frame[containsverbatim]{
	\frametitle{Les ent{\^e}tes et pieds de page}
	Pour les ent{\^e}tes et pieds de page, on utilise les commandes suivantes :
	\begin{lstlisting}
		\lhead{Ajout d'un ent{\^e}te {\`a} gauche} 
		\chead{Ajout d'un ent{\^e}te au centre}
		\rhead{Ajout d'un ent{\^e}te {\`a} droite}
		\lfoot{Ajout d'un pied de page {\`a} gauche} 
		\cfoot{Ajout d'un pied de page au centre} 
		\rfoot{Ajout d'un pied de page {\`a} droite}
	\end{lstlisting}
}

\frame[containsverbatim]{
	\frametitle{Les notes de page}
	Les notes de page peuvent être ajoutée de la manière suivante :
	\begin{lstlisting}	
	Dijkstra\footnote{math\'{e}maticien et informaticien n\'{e}erlandais du XX$^{e}$ si\`{e}cle} avait une tr\`{e}s belle \'{e}criture manuscrite et a toujours refus\'{e} d'utiliser un traitement de texte, malgr\'{e} son domaine d'activit\'{e}, pr\'{e}f\'{e}rant la lettre manuscrite photocopi\'{e}e.
	\end{lstlisting}
	
	Dijkstra$^{[1]}$ avait une tr\`{e}s belle \'{e}criture manuscrite et a toujours refus\'{e} d'utiliser un traitement de texte, malgr\'{e} son domaine d'activit\'{e}, pr\'{e}f\'{e}rant la lettre manuscrite photocopi\'{e}e.
	
	--
	
	[1] : math\'{e}maticien et informaticien n\'{e}erlandais du XX$^{e}$ si\`{e}cle
}

\frame[containsverbatim]{
	\frametitle{Les listes}
	Les listes se construisent ainsi :
	\begin{LTXexample}[pos=r]
	\begin{enumerate}
	\item partie 1
	\item partie 2
	\item partie 3
	\end{enumerate}
	\begin{itemize}
	\item partie 1
	\item partie 2
	\item partie 3
	\end{itemize}
	\begin{description}
	\item[cas 1.0] partie 1
	\item[cas 1.1] partie 2
	\item[cas 2.0] partie 3
	\end{description}
	\end{LTXexample}
}

\frame[containsverbatim]{
	\frametitle{Les tableaux}
	Pour les tableaux, il faut s'y prendre de la manière suivante :
	\begin{lstlisting}
		\begin{table}
			\begin{tabular}{|l|cc|}
				OS & Plateforme & Part des serveurs http \\
				\hline
				Unix & Toutes & 32\% \\
				Linux & Toutes & 26\% \\
				Windows NT & Intel & 23\% \\
			\end{tabular}
			\caption{Ceci est un tableau pr\'esentant la part des serveurs occup\'es par chaque syst\`eme d'exploitation.}\label{tab_serveur}
		\end{table}
		Ici, je fais r\'ef\'erence \`a mon tableau \ref{tab_serveur}
	\end{lstlisting}
}

\frame[containsverbatim]{
	\frametitle{Les tableaux}
	
	\begin{table}
		\begin{tabular}{|l|cc|}
			OS & Plateforme & Part des serveurs http \\
			\hline
			Unix & Toutes & 32\% \\
			Linux & Toutes & 26\% \\
			Windows NT & Intel & 23\% \\
		\end{tabular}
		\caption{Ceci est un tableau pr\'esentant la part des serveurs occup\'es par chaque syst\`eme d'exploitation.}\label{tab_serveur}
	\end{table}
	Ici, je fais r\'ef\'erence \`a mon tableau \ref{tab_serveur}
}

\frame[containsverbatim]{
\frametitle{Le type du document}

La commande documentclass spécifie le type de document et son mode de rendu.
Il est possible de spécifier le rendu d'un document lorsque l'on spécifie le type de document :

\textbackslash documentclass[options]\{type\}

Les types de document :
   \begin{itemize}
    \item article : articles scientifiques, présentations, rapports courts, etc.
    \item proc : procédures
    \item minimal : minimaliste
    \item report : longs rapports
    \item book : livres
    \item beamer : présentations type powerpoint.
   \end{itemize}
}

\frame[containsverbatim]{
\frametitle{Le type du document}

\textbackslash documentclass[options]\{type\}

Les options du document :
   \begin{itemize}
    \item 10pt, 11pt, 12pt : taille de la police de caractères
    \item a4paper, letterpaper, ... : format du papier
    \item lanscape : orientation paysage
    \item etc.   
   \end{itemize}
}

\frame[containsverbatim]{
\frametitle{Le type du document}

Exemple :

\begin{LTXexample}
\documentclass[11pt,a4paper,fleqn,notitlepage,twocolumn]{article}
\begin{document}
Contenu
\end{document}
\end{LTXexample}
}

\frame[containsverbatim]{
	\frametitle{La bibliographie}
	La bibliographie sera très utile pour indiquer vos références et permettre à vos lecteurs de les vérifier et prolonger leur lecture.
	
	\begin{LTXexample}
		\begin{thebibliography}{9}
			\bibitem{wikipedia}
			Wikipedia, somme de toutes les connaissances humaines.
		\end{thebibliography}
		
		On trouve des informations tr{\`e}s s{\'e}rieuses si l'on cherche au bon endroit[\cite{wikipedia}].
	\end{LTXexample}
}

\section{Packages}

\subsection{Packages courants}

\frame[containsverbatim]{
\frametitle{Les packages}

La commande usepackage spécifie l'utilisation d'un package.

\textbackslash usepackage[options]{nom du package}

Quelques packages utilisé couramment :

   \begin{itemize}
    \item \textbackslash usepackage[francais, turkish]\{babel\}
    \item \textbackslash usepackage\{tikz\}
    \item \textbackslash usepackage[all]\{xy\}
    \item \textbackslash usepackage\{amsmath\}
    \item \textbackslash usepackage\{graphicx\}
   \end{itemize}
}

\subsection{Babel}
\frame[containsverbatim]{
\frametitle{Babel}
Babel est un package qui permet de générer des documents en plusieurs langues.
\begin{LTXexample}
\selectlanguage{turkish}
Bu T{\"{u}}rk\c cedir,

\selectlanguage{francais}
Et c'est du fran\c cais
\end{LTXexample}

En savoir plus : \cite{babel}
}

\subsection{Tikz}
\frame[containsverbatim]{
\frametitle{Tikz}
TikZ permet de dessiner des schémas.

\begin{LTXexample}
\begin{tikzpicture}
\draw (0,0) circle (1) ;
\end{tikzpicture}
\end{LTXexample}

Remarque : il est aisé de générer des schémas TikZ avec Inkscape.

En savoir plus : \cite{tikz1,tikz2}
}

\subsection{xy}
\frame[containsverbatim]{
\frametitle{xy}
xy permet d'afficher des diagrammes.
\begin{LTXexample}
\begin{displaymath}
    \xymatrix{ A \ar[r] & B \ar[d] \\
               D \ar[u] & C \ar[l] }
\end{displaymath}
\end{LTXexample}
}

\frame[containsverbatim]{
\frametitle{xy}
Un exemple plus complexe :
\begin{LTXexample}
\xymatrix{
    A \ar@{^{(}->}[r]^f \ar@{.>}[d]_\phi \ar@{=>}[rd] & B \ar@{=}[d]^\psi \\
    C \ar@{->>}[r]_g & D
  }
\end{LTXexample}
}

\frame[containsverbatim]{
\frametitle{xy}
Un dernier exemple :
\begin{LTXexample}
\xymatrix{
  A \ar[r]^u_d \ar[rd]^u_d &
  B \ar[d]^u_d &
  C \ar[l]^u_d \ar[ld]^u_d\\
  &D}
\end{LTXexample}

Pour en savoir plus : \cite{asmth}
}

\subsection{amsmath}
\frame[containsverbatim]{
\frametitle{amsmath}
amsmath fournit de nombreuses fonctionnalités pour afficher des équations ou des structures mathématiques.
\begin{LTXexample}
\begin{align}
a_1& =b_1+c_1\\
a_2& =b_2+c_2-d_2+e_2
\end{align}
\end{LTXexample}

Pour en savoir plus : \cite{asmth}
}

\frame[containsverbatim]{
\frametitle{amsmath}
amsmath permet également de générer des matrices.
\begin{LTXexample}
$A_{m,n} =
 \begin{pmatrix}
  a_{1,1} & a_{1,2} & \cdots & a_{1,n} \\
  a_{2,1} & a_{2,2} & \cdots & a_{2,n} \\
  \vdots  & \vdots  & \ddots & \vdots  \\
  a_{m,1} & a_{m,2} & \cdots & a_{m,n}
 \end{pmatrix}$
\end{LTXexample}

Pour en savoir plus : \cite{latex_math}
}

\subsection{amsthm}
\frame[containsverbatim]{
\frametitle{amsthm}
amsthm étend la fonction \textbackslash newtheorem.
\begin{LTXexample}
\newtheorem{lem}{Th{\'e}or{\`e}me}
\begin{lem} 
Ceci est mon th{\'e}or{\`e}me. 
\end{lem}
\end{LTXexample}

Pour en savoir plus : \cite{latex_math}
}

\subsection{graphicx}
\frame[containsverbatim]{
\frametitle{graphicx}
graphicx permet d'afficher des images.
\begin{LTXexample}
\begin{figure}[position]
   \includegraphics[scale=0.10]{./Images/latex.jpg}
\end{figure}
\end{LTXexample}

En savoir plus : \cite{images}
}

\section{Bibliographie}
\frame[containsverbatim]{
	\frametitle{Bibliographie}
	Bibliographie
	
	\begin{thebibliography}{9}
		\bibitem{cours} http://pbil.univ-lyon1.fr/members/mbailly/Comm\_Scientifique/M1/cours\_latex.pdf
		\bibitem{td} http://www.math.ens.fr/~millien/tdlatex/poly.pdf
		\bibitem{latex_math} http://en.wikibooks.org/wiki/LaTeX/Mathematics
		\bibitem{asmth} http://mirror.isoc.org.il/pub/ctan/macros/latex/required/amslatex/amscls/doc/amsthdoc.pdf
		\bibitem{babel} http://www.tuteurs.ens.fr/logiciels/latex/manuel.html
		\bibitem{tikz1} http://math.et.info.free.fr/TikZ/bdd/TikZ-Impatient.pdf
		\bibitem{tikz2} http://www.texample.net/tikz/examples/
		\bibitem{images} http://fr.wikibooks.org/wiki/LaTeX/Inclure\_des\_images
	\end{thebibliography}
}

\end{document}