\makeatletter\let\ifGm@compatii\relax\makeatother
\documentclass{beamer}
\usepackage{etex}

\usepackage[utf8]{inputenc}
%\usepackage[T1]{fontenc}
\usepackage[francais, turkish]{babel}
%\usepackage{amssymb,amsthm,amsmath}
\usepackage{xcolor}
\usepackage{tikz}
\usepackage[all]{xy}

\usepackage{listings}
%\usepackage{verbatim}

\usepackage{bera}
\usepackage{caption}
\captionsetup{%
   margin=0em,
   font={scriptsize,rm},
   labelfont={color=Maroon,bf},
   justification=justified,%default: RaggedRight. Other options: justified, centering
   labelsep=quad%default:colon. Options: period, space, quad, newline
}
\usepackage{showexpl}
\usepackage{etoolbox}
\makeatletter
\patchcmd{\SX@codeInput}{xleftmargin=0pt,xrightmargin=0pt}{}
  {\typeout{***Successfully patched \protect\SX@codeInput***}}
  {\typeout{***ERROR! Failed to patch \protect\SX@codeInput***}}
\makeatother

\lstloadlanguages{[LaTeX]Tex} 
\lstset{% 
    basicstyle=\ttfamily\small, 
    commentstyle=\itshape\ttfamily\small, 
    showspaces=false, 
    showstringspaces=false, 
    breaklines=true, 
    breakautoindent=true, 
    captionpos=t 
} 
\lstset{%
    breaklines=true,%default : false 
    breakindent=10pt,%default: 20pt 
    linewidth=\linewidth,%default : \linewidth,
    basicstyle=\ttfamily\tiny,% cannot take arguments
    keywordstyle=\color{Blue}\sffamily\bfseries,                                
    identifierstyle=\color{Black},                                      
    commentstyle=\color{OliveGreen}\itshape,                                    
    stringstyle=\rmfamily,                                                      
    showspaces=false,%default false
    showstringspaces=false,%default: true
    backgroundcolor=\color{Yellow!30},
    frame=single,%default frame=none 
    rulecolor=\color{Red},  
    % the following must be defined to make hacking work.
    framerule=0.4pt,%expand outward 
    framesep=3pt,%expand outward
    xleftmargin=3.4pt,%to make the frame fits in the text area. 
    xrightmargin=3.4pt,%to make the frame fits in the text area. 
    tabsize=2%,%default :8 only influence the lstlisting and lstinline.
}
\lstset{%
    %explpreset below will affect LTX only.
    explpreset={%
                            language={[LaTeX]TeX},
                            captionpos=t,
                            aboveskip=2\fboxsep,                        
                            pos=b,
                            vsep=\fboxsep%vertical space between formatted and input.
                         }%
}

\usetheme{Hannover}

\newtheorem{defi}{Définition}
\newtheorem{thm}{Théorème}

\title{Introduction au \LaTeX}
\author{Charles Martin}

\begin{document}

\definecolor{Red}{RGB}{255,0,0}
\definecolor{Grey}{RGB}{205,201,201}
\definecolor{Yellow}{RGB}{255,255,0}
\definecolor{Blue}{RGB}{50,50,150}
\definecolor{Black}{RGB}{0,0,0}
\definecolor{OliveGreen}{RGB}{0,100,0}

\frame{\titlepage}

%\setcounter{tocdepth}{1}
%\begin{frame}{Programme}
% \tableofcontents[pausesections]
%\end{frame}

\section{\TeX}	
\subsection{Introduction}
\begin{frame}{\TeX}
  Le moteur derrière \LaTeX est \TeX .
  
  \TeX a été crée en 1978 par Donald Knuth.
  \begin{block}{Les objectifs :}    
   \begin{itemize}
    \item Permettre d’écrire des livres de qualité avec un minimum d’effort
    \item Fournir un système qui donnera toujours le même résultat, maintenant et à l’avenir.
   \end{itemize}
  \end{block}
\end{frame}

\subsection{Syntaxe}
\begin{frame}{Hello World}
	Exemple de Hello World en \TeX :
    \lstinputlisting{Sources/HelloWorldTex.tex}
\end{frame}

\section{\LaTeX}
\subsection{Introduction}
\begin{frame}{\LaTeX}
  \LaTeX a été crée en 1980 par Leslie Lamport.
  \begin{block}{Les objectifs :}    
   \begin{itemize}
    \item Fournir un language de haut niveau.
    \item Utiliser la puissance de TeX.
   \end{itemize}
  \end{block}
\end{frame}

\subsection{Syntaxe}

\frame[containsverbatim]{
\frametitle{Hello World}
Exemple de Hello World en \LaTeX :
\begin{LTXexample}
\documentclass{article}
\begin{document}
Hello, World !
\end{document}
\end{LTXexample}
}

\subsection{Les espaces}

\frame[containsverbatim]{
\frametitle{Les espaces}

\begin{LTXexample}
Le nombre       d'espaces n'est pas important.
\end{LTXexample}

\begin{LTXexample}
Une ligne vide apparait.

Un nouveau paragraphe commence.
\end{LTXexample}
}

\subsection{Les caractères spéciaux}

\frame[containsverbatim]{
\frametitle{Les caractères spéciaux}
Les symboles suivants sont des caractères réservés :

\# \$ \% \^{} \& \_ \{ \} \~{} \textbackslash

Pour les utiliser, placez un \textbackslash {} devant:

\begin{LTXexample}
\# \$ \% \^{} \& \_ \{ \} \~{}
\textbackslash
\end{LTXexample}
}

\subsection{Les commandes}

\frame[containsverbatim]{
\frametitle{Les commandes}
Une commande commence par un \textbackslash{} et a un nom constitué de lettres uniquement.

\textbackslash command[parametre optionnel]\{parametre\}

\begin{LTXexample}
Vous pouvez \textsl{compter} sur moi!
\end{LTXexample}

\begin{LTXexample}
Vous pouvez commencer une nouvelle ligne juste ici ! \newline Merci!
\end{LTXexample}
}

\subsection{Les commentaires}

\frame[containsverbatim]{
\frametitle{Les commentaires}
Les commentaires commencent par le caractère \%.
Un commentaire ne sera jamais affiché, et servira généralement à ajouter des remarques sur le code, voir à désactiver du code.
\begin{LTXexample}
Ce message est visible.
%Mais pas ce message.
%\textsl{Ni cette commande.}
\end{LTXexample}
}

\subsection{La structure du fichier}

\frame[containsverbatim]{
\frametitle{La structure du fichier}
Lorsque \LaTeX{} traite un fichier, il s'attend à suivre une certaine structure.

Cette structure est composée :

   \begin{itemize}
    \item du type de document
    \item du titre
    \item de l'auteur
    \item de la date
    \item du contenu
   \end{itemize}
}

\frame[containsverbatim]{
\frametitle{La structure d'un article}
Exemple de structure d'un article :
\begin{LTXexample}
%On specifie le type de document, ici, un article
\documentclass{article} 
%On specifie le titre de l'article
\title{Document LaTeX} 
%On specifie l'auteur
\author{Charles Martin} 
%On specifie la date
\date{Novembre 2014} 
%Le document commence ici
\begin{document} 
%On trouvera ici le contenu
Contenu du document.
%Le document se termine ici 
\end{document} 
\end{LTXexample}
}

\frame[containsverbatim]{
\frametitle{La structure d'une pr\'esentation type powerpoint}
Exemple de structure d'un article :
\begin{LTXexample}
%On specifie le type de document, ici, un article
\documentclass{beamer} 
%On choisit un theme
\usetheme{Warsaw}
%On specifie le titre de l'article
\title{Pr\'esentation LaTeX} 
%On specifie l'auteur
\author{Charles Martin} 
%On specifie la date
\date{Novembre 2014} 
%Le document commence ici
\begin{document} 
%\begin{frame}
%\frametitle{Titre}
%\framesubtitle{Sous-Titre}
%Contenu
%\end{frame}
\end{document} 
\end{LTXexample}
}

\subsubsection{Le type de document}

\frame[containsverbatim]{
\frametitle{Le type du document}

La commande documentclass spécifie le type de document et son mode de rendu.
Il est possible de spécifier le rendu d'un document lorsque l'on spécifie le type de document :

\textbackslash documentclass[options]\{type\}

Les types de document :
   \begin{itemize}
    \item article : articles scientifiques, présentations, rapports courts, etc.
    \item proc : procédures
    \item minimal : minimaliste
    \item report : longs rapports
    \item book : livres
    \item beamer : présentations type powerpoint.
   \end{itemize}
}

\frame[containsverbatim]{
\frametitle{Le type du document}

\textbackslash documentclass[options]\{type\}

Les options du document :
   \begin{itemize}
    \item 10pt, 11pt, 12pt : taille de la police de caractères
    \item a4paper, letterpaper, ... : format du papier
    \item lanscape : orientation paysage
    \item etc.   
   \end{itemize}
}

\frame[containsverbatim]{
\frametitle{Le type du document}

Exemple :

\begin{LTXexample}
\documentclass[11pt,a4paper,fleqn,notitlepage,twocolumn]{article}
\begin{document}
Contenu
\end{document}
\end{LTXexample}
}

\section{Packages}

\subsection{Packages courants}

\frame[containsverbatim]{
\frametitle{Les packages}

La commande usepackage spécifie l'utilisation d'un package.

\textbackslash usepackage[options]{nom du package}

Les packages les plus courants :

   \begin{itemize}
    
    \item babel
    \item tikz
    \item xy
    \item amsthm
    \item amssymb
    \item graphicx
   \end{itemize}
}

%http://www.tuteurs.ens.fr/logiciels/latex/manuel.html
\subsection{Babel}
\frame[containsverbatim]{
\frametitle{Babel}
Babel est un package qui permet de générer des documents en plusieurs langues.
\begin{LTXexample}
\selectlanguage{turkish}
Bu T{\"{u}}rk\c cedir,

\selectlanguage{francais}
Et c'est du fran\c cais
\end{LTXexample}
}

%http://math.et.info.free.fr/TikZ/bdd/TikZ-Impatient.pdf
%http://www.texample.net/tikz/examples/
\subsection{Tikz}
\frame[containsverbatim]{
\frametitle{Tikz}
TikZ permet de dessiner des schémas.

\begin{LTXexample}
\begin{tikzpicture}
\draw (0,0) circle (1) ;
\end{tikzpicture}
\end{LTXexample}

Remarque : il est aisé de générer des schémas TikZ avec Inkscape.
}

%http://en.wikibooks.org/wiki/LaTeX/Xy-pic
%xy : xy is a special package for drawing diagrams.
\subsection{xy}
\frame[containsverbatim]{
\frametitle{xy}
xy permet d'afficher des matrices.
\begin{LTXexample}
\begin{displaymath}
    \xymatrix{ A \ar[r] & B \ar[d] \\
               D \ar[u] & C \ar[l] }
\end{displaymath}
\end{LTXexample}
}


%    \item amsthm
%    \item amssymb
%    \item graphicx

\end{document}