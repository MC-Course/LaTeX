\makeatletter\let\ifGm@compatii\relax\makeatother
\documentclass{beamer}

\usepackage[utf8]{inputenc}
\usepackage[T1]{fontenc}
\usepackage[francais]{babel}
\usepackage{amssymb,amsthm,amsmath}
\usepackage{xcolor}

\usepackage{listings}
\usepackage{verbatim}

\usepackage{xcolor}
\usepackage{bera}
\usepackage{caption}
\captionsetup{%
   margin=0em,
   font={scriptsize,rm},
   labelfont={color=Maroon,bf},
   justification=justified,%default: RaggedRight. Other options: justified, centering
   labelsep=quad%default:colon. Options: period, space, quad, newline
}
\usepackage{showexpl}
\usepackage{etoolbox}
\makeatletter
\patchcmd{\SX@codeInput}{xleftmargin=0pt,xrightmargin=0pt}{}
  {\typeout{***Successfully patched \protect\SX@codeInput***}}
  {\typeout{***ERROR! Failed to patch \protect\SX@codeInput***}}
\makeatother

\lstloadlanguages{[LaTeX]Tex} 
\lstset{% 
    basicstyle=\ttfamily\small, 
    commentstyle=\itshape\ttfamily\small, 
    showspaces=false, 
    showstringspaces=false, 
    breaklines=true, 
    breakautoindent=true, 
    captionpos=t 
} 
\lstset{%
    %explpreset below will affect LTX only.
    explpreset={%
                            language={[LaTeX]TeX},
                            captionpos=t,
                            aboveskip=2\fboxsep,                        
                            pos=b,
                            vsep=\fboxsep%vertical space between formatted and input.
                         }%
}

\usetheme{Hannover}

\newtheorem{defi}{Définition}
\newtheorem{thm}{Théorème}

\title{Introduction au \LaTeX}
\author{Charles Martin}

\begin{document}

\frame{\titlepage}

%\setcounter{tocdepth}{1}
%\begin{frame}{Programme}
% \tableofcontents[pausesections]
%\end{frame}

\section{\TeX}	
\subsection{Introduction}
\begin{frame}{\TeX}
  Le moteur derrière \LaTeX est \TeX .
  
  \TeX a été crée en 1978 par Donald Knuth.
  \begin{block}{Les objectifs :}    
   \begin{itemize}
    \item Permettre d’écrire des livres de qualité avec un minimum d’effort
    \item Fournir un système qui donnera toujours le même résultat, maintenant et à l’avenir.
   \end{itemize}
  \end{block}
\end{frame}

\subsection{Syntaxe}
\begin{frame}{Hello World}
	Exemple de Hello World en \TeX :
    \lstinputlisting{Sources/HelloWorldTex.tex}
\end{frame}

\section{\LaTeX}
\subsection{Introduction}
\begin{frame}{\LaTeX}
  \LaTeX a été crée en 1980 par Leslie Lamport.
  \begin{block}{Les objectifs :}    
   \begin{itemize}
    \item Fournir un language de haut niveau.
    \item Utiliser la puissance de TeX.
   \end{itemize}
  \end{block}
\end{frame}

\subsection{Syntaxe}

\frame[containsverbatim]{
\frametitle{Hello World}
Exemple de Hello World en \LaTeX :
\begin{LTXexample}
\documentclass{article}
\title{Cours de LaTeX}
\author{Charles Martin}
\date{Novembre 2014}
\begin{document}
Hello, World !
\end{document}
\end{LTXexample}
}

\frame[containsverbatim]{
\frametitle{Les espaces}

\begin{LTXexample}
Le nombre       d'espaces n'est pas important.
\end{LTXexample}

\begin{LTXexample}
Une ligne vide apparait.

Un nouveau paragraphe commence.
\end{LTXexample}
}

\frame[containsverbatim]{
\frametitle{Les caractères spéciaux}
Les symboles suivants sont des caractères réservés :

\# \$ \% \^{} \& \_ \{ \} \~{} \textbackslash

Pour les utiliser, placez un \textbackslash {} devant:

\begin{LTXexample}
\# \$ \% \^{} \& \_ \{ \} \~{}
\textbackslash
\end{LTXexample}
}

\frame[containsverbatim]{
\frametitle{Les commandes}
Une commande commence par un \textbackslash{} et a un nom constitué de lettres uniquement.

\textbackslash command[parametre optionnel]\{parametre\}

\begin{LTXexample}
Vous pouvez \textsl{compter} sur moi!
\end{LTXexample}

\begin{LTXexample}
Vous pouvez commencer une nouvelle ligne juste ici ! \newline Merci!
\end{LTXexample}
}

\frame[containsverbatim]{
\frametitle{Les commentaires}
Les commentaires commencent par le caractère \%.
Un commentaire ne sera jamais affiché, et servira généralement à ajouter des remarques sur le code, voir à désactiver du code.
\begin{LTXexample}
Ce message est visible.
%Mais pas ce message.
%\textsl{Ni cette commande.}
\end{LTXexample}
}

\frame[containsverbatim]{
\frametitle{La structure du fichier}
Lorsque \LaTeX{} traite un fichier, il s'attend à suivre une certaine structure.
Chaque source \LaTeX{} commence avec la commande :
\newline
\newline
\textbackslash documentclass \{ \}
\newline
\newline
On spécifie ici quel type de document l'on va écrire.

%\begin{LTXexample}
%\documentclass{article} %On va écrire un article
%\title{Cours de LaTeX} %On spécifie le titre
%\author{Charles Martin} %On spécifie l'auteur
%\date{Novembre 2014} %On spécifie la date
%\begin{document} %Le document commence ici
%Contenu %On trouvera ici le contenu
%\end{document} %Le document se termine ici
%\end{LTXexample}

}

\end{document}