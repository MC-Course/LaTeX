\makeatletter\let\ifGm@compatii\relax\makeatother
\documentclass{beamer}

\usepackage[utf8]{inputenc}
\usepackage[T1]{fontenc}
\usepackage[francais]{babel}
\usepackage{amssymb,amsthm,amsmath}
\usepackage{xcolor}

\usepackage{listings}
\usepackage{verbatim}
\usepackage{showexpl}

\lstloadlanguages{[LaTeX]Tex} 
\lstset{% 
    basicstyle=\ttfamily\small, 
    commentstyle=\itshape\ttfamily\small, 
    showspaces=false, 
    showstringspaces=false, 
    breaklines=true, 
    breakautoindent=true, 
    captionpos=t 
} 

\usetheme{Hannover}

\newtheorem{defi}{Définition}
\newtheorem{thm}{Théorème}

\title{Introduction au \LaTeX}
\author{Charles Martin}

\begin{document}

\frame{\titlepage}

%\setcounter{tocdepth}{1}
%\begin{frame}{Programme}
% \tableofcontents[pausesections]
%\end{frame}

\section{\TeX}	
\subsection{Introduction}
\begin{frame}{\TeX}
  Le moteur derrière \LaTeX est \TeX .
  
  \TeX a été crée en 1978 par Donald Knuth.
  \begin{block}{Les objectifs :}    
   \begin{itemize}
    \item Permettre d’écrire des livres de qualité avec un minimum d’effort
    \item Fournir un système qui donnera toujours le même résultat, maintenant et à l’avenir.
   \end{itemize}
  \end{block}
\end{frame}

\subsection{Syntaxe}
\begin{frame}{Hello World}
	Exemple de Hello World en \TeX :
    \lstinputlisting{Sources/HelloWorldTex.tex}
\end{frame}

\section{\LaTeX}
\subsection{Introduction}
\begin{frame}{\LaTeX}
  \LaTeX a été crée en 1980 par Leslie Lamport.
  \begin{block}{Les objectifs :}    
   \begin{itemize}
    \item Fournir un language de haut niveau.
    \item Utiliser la puissance de TeX.
   \end{itemize}
  \end{block}
\end{frame}

\subsection{Syntaxe}

\begin{frame}{Hello World}
	Exemple de Hello World en \LaTeX :
    \lstinputlisting{Sources/HelloWorld.tex}
\end{frame}


\begin{frame}{Les espaces}
\end{frame}
\end{document}