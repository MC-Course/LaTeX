\makeatletter\let\ifGm@compatii\relax\makeatother
\documentclass{beamer}

\usepackage[utf8]{inputenc}
\usepackage[T1]{fontenc}
\usepackage[francais]{babel}

\usepackage{amssymb,amsthm,amsmath}

\usepackage{xcolor}

\usetheme{Warsaw}

\newtheorem{defi}{Définition}
\newtheorem{thm}{Théorème}

\title{Exemple de présentation Beamer}
\author{LM204}


\begin{document}

\frame{\titlepage}

\setcounter{tocdepth}{1}
\begin{frame}{Programme}
 \tableofcontents[pausesections]
\end{frame}

\section{Introduction générale}

\begin{frame}{Le problème de base}
  On doit présenter un certain résultat lors d'une conférence.
  \pause
  \begin{block}{Cahier des charges}
    La présentation doit \^etre:
   \begin{itemize}
    \item claire
    \item concise
    \item vidéo-projetée
   \end{itemize}
  \end{block}
\end{frame}

\begin{frame}{La solution}
  \pause
  \begin{center}
   \Huge Beamer!
  \end{center}
\end{frame}

\section{Le résultat}

\begin{frame}
 \tableofcontents[currentsection]
\end{frame}

\subsection{Modèle}

\begin{frame}{Où l'on décrit le modèle}
 Sans fioritures.
 \pause
 
 Sans doutes quelques définitions?
\end{frame}

\begin{frame}{Où l'on décrit le modèle}
 \begin{defi}
  On définit ici l'objet.
 \end{defi}
\end{frame}

\subsection{Théorème}

\begin{frame}{Le résultat}
  \begin{thm}[principe fondamental de notre travail]
   \'Enoncé du théorème
  \end{thm}
  \pause
  \begin{proof}
   La preuve du résultat:\pause
   \begin{enumerate}[<+->]
    \item en
    \item trois
    \item étapes \qedhere
   \end{enumerate}
  \end{proof}
\end{frame}

\section{Pour aller plus loin}

\begin{frame}
 \tableofcontents[currentsection]
\end{frame}

\begin{frame}{Amusons-nous}
  \alt<11-12>{
  \begin{alertblock}{STOP!}
   \invisible<11>{Attention à ne pas trop en faire!}
  \end{alertblock}
  }{
  Beamer permet de régler l'apparition des éléments de la diapositive de plusieurs façons:
  \visible<2-5,10>{
  \begin{itemize}
   \item<2-> dans
   \item<4-> n'importe quel
   \item<3-> \only<3>{l' }ordre.
  \end{itemize}
  }
  \uncover<5->{
  \visible<9->{ou m\^eme avant!}
  \begin{enumerate}
   \item en les mettant \textbf<5>{en gras},
   \item en \textcolor<6>{blue}{couleur},
   \item<alert@7> ou en valeur\alt<8-9>{,}{.}
   \item<8-9> pendant un court laps de temps.
  \end{enumerate}
  }
  \visible<10>{Et (tout?) réafficher!}
  }
\end{frame}

\end{document}